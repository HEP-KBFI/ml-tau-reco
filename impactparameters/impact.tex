\documentclass[a4paper,english,11pt]{article}
\usepackage[bindingoffset=0.5cm,left=2.5cm,right=2.5cm,top=2.5cm,bottom=2.5cm,footskip=1.0cm]{geometry}

% Prevent breaking of footnotes to pages that are different from the page where the footnote symbol appears
\interfootnotelinepenalty=10000

% Useful packages
\usepackage{amsmath}
\usepackage{amssymb}
\usepackage{xspace}
%\usepackage{heppennames}
\usepackage{float}
\usepackage{hyperref}
\usepackage{subcaption}
\usepackage{relsize}
\usepackage{multirow}
\usepackage{xcolor}
\usepackage{graphicx}
\usepackage{lineno}
\usepackage[english]{babel}
\usepackage[bb=dsserif]{mathalpha}
\usepackage{bbding}
%\usepackage{hepparticles}

\usepackage{authblk}
\bibliographystyle{elsarticle-num}


\newcommand{\Pagne}{\ensuremath{\bar{\nu}_{\textrm{e}}}\xspace}
\newcommand{\Pagngm}{\ensuremath{\bar{\nu}_{\mu}}\xspace}
\newcommand{\Pagngt}{\ensuremath{\bar{\nu}_{\tau}}\xspace}
\newcommand{\Paq}{\ensuremath{\bar{\textrm{q}}}\xspace}
\newcommand{\Paqb}{\ensuremath{\bar{\textrm{b}}}\xspace}
\newcommand{\Paqc}{\ensuremath{\bar{\textrm{c}}}\xspace}
\newcommand{\PcB}{\ensuremath{\textrm{B}_{\textrm{c}}^{+}}\xspace}
\newcommand{\Pe}{\ensuremath{\textrm{e}}\xspace}
\newcommand{\Pem}{\ensuremath{\textrm{e}^{-}}\xspace}
\newcommand{\Pep}{\ensuremath{\textrm{e}^{+}}\xspace}
\newcommand{\Pgg}{\ensuremath{\gamma}\xspace}
\newcommand{\Pggx}{\ensuremath{\gamma^{\textasteriskcentered}}\xspace}
\newcommand{\Pgm}{\ensuremath{\mu}\xspace}
\newcommand{\Pgn}{\ensuremath{\nu}\xspace}
\newcommand{\Pgne}{\ensuremath{\nu_{\textrm{e}}}\xspace}
\newcommand{\Pgngm}{\ensuremath{\nu_{\mu}}\xspace}
\newcommand{\Pgngt}{\ensuremath{\nu_{\tau}}\xspace}
\newcommand{\Pgpm}{\ensuremath{\pi^{-}}\xspace}
\newcommand{\Pgpp}{\ensuremath{\pi^{+}}\xspace}
\newcommand{\Pgppm}{\ensuremath{\pi^{\pm}}\xspace}
\newcommand{\Pgpz}{\ensuremath{\pi^{\textrm{0}}}\xspace}
\newcommand{\Pgt}{\ensuremath{\tau}\xspace}
\newcommand{\Pgtm}{\ensuremath{\tau^{-}}\xspace}
\newcommand{\Pgtp}{\ensuremath{\tau^{+}}\xspace}
\newcommand{\Pgtpm}{\ensuremath{\tau^{\pm}}\xspace}
\newcommand{\PH}{\ensuremath{\textrm{H}}\xspace}
\newcommand{\PHm}{\ensuremath{\textrm{H}^{-}}\xspace}
\newcommand{\PHp}{\ensuremath{\textrm{H}^{+}}\xspace}
\newcommand{\PHpm}{\ensuremath{\textrm{H}^{\pm}}\xspace}
\newcommand{\PK}{\ensuremath{\textrm{K}}\xspace}
\newcommand{\PKm}{\ensuremath{\textrm{K}^{-}}\xspace}
\newcommand{\PKp}{\ensuremath{\textrm{K}^{+}}\xspace}
\newcommand{\PKpm}{\ensuremath{\textrm{K}^{\pm}}\xspace}
\newcommand{\PKz}{\ensuremath{\textrm{K}^{\textrm{0}}}\xspace}
\newcommand{\PKzL}{\ensuremath{\textrm{K}_{\textrm{L}}^{\textrm{0}}}\xspace}
\newcommand{\PKzS}{\ensuremath{\textrm{K}_{\textrm{S}}^{\textrm{0}}}\xspace}
\newcommand{\Pl}{\ensuremath{\ell}\xspace}
\newcommand{\Pp}{\ensuremath{\textrm{p}}\xspace}
\newcommand{\Pq}{\ensuremath{\textrm{q}}\xspace}
\newcommand{\Pqb}{\ensuremath{\textrm{b}}\xspace}
\newcommand{\Pqc}{\ensuremath{\textrm{c}}\xspace}
\newcommand{\PW}{\ensuremath{\textrm{W}}\xspace}
\newcommand{\PWm}{\ensuremath{\textrm{W}^{-}}\xspace}
\newcommand{\PWp}{\ensuremath{\textrm{W}^{+}}\xspace}
\newcommand{\PWpm}{\ensuremath{\textrm{W}^{\pm}}\xspace}
\newcommand{\PZ}{\ensuremath{\textrm{Z}}\xspace}

%
% please place your own definitions here and don't use \def but
% \newcommand{}{}
%
\newcommand{\tauh}{\ensuremath{\Pgt_{\textrm{h}}}\xspace}
\newcommand{\h}{\ensuremath{\textrm{h}}\xspace}
\newcommand{\hpm}{\ensuremath{\textrm{h}^{\pm}}\xspace}
\newcommand{\hplus}{\ensuremath{\textrm{h}^{+}}\xspace}
\newcommand{\hminus}{\ensuremath{\textrm{h}^{-}}\xspace}
\newcommand{\hzero}{\ensuremath{\textrm{h}^{\textrm{0}}}\xspace}
\newcommand{\aone}{\ensuremath{\textrm{a}_{\textrm{1}}}\xspace}
\newcommand{\pT}{\ensuremath{p_{\textrm{T}}}\xspace}
\newcommand{\pTi}{\ensuremath{p_{\textrm{T},i}}\xspace}
\newcommand{\pTj}{\ensuremath{p_{\textrm{T},j}}\xspace}
\newcommand{\kt}{\ensuremath{k_{\textrm{t}}}\xspace}
\newcommand{\ptmiss}{\ensuremath{p_{\textrm{T}}^\textrm{miss}}\xspace}
\newcommand{\MeV}{\ensuremath{\textrm{MeV}}\xspace}
\newcommand{\GeV}{\ensuremath{\textrm{GeV}}\xspace}
\newcommand{\TeV}{\ensuremath{\textrm{TeV}}\xspace}
\newcommand{\sig}{\ensuremath{\textrm{sig}}\xspace}
\newcommand{\bgr}{\ensuremath{\textrm{bgr}}\xspace}
\newcommand{\gen}{\ensuremath{\textrm{gen}}\xspace}
\newcommand{\rec}{\ensuremath{\textrm{rec}}\xspace}
\newcommand{\jet}{\ensuremath{\textrm{jet}}\xspace}
\newcommand{\PCA}{\ensuremath{\textrm{PCA}}\xspace}
\newcommand{\PV}{\ensuremath{\textrm{PV}}\xspace}
\newcommand{\trk}{\ensuremath{\textrm{trk}}\xspace}
\newcommand{\refpt}{\ensuremath{\textrm{ref}}\xspace}
\newcommand{\jetR}{\ensuremath{\langle r \rangle}\xspace}
\newcommand{\jetM}{\ensuremath{M_{\jet}}\xspace}
\newcommand{\Dtau}{\ensuremath{\mathcal{D}_{\Pgt}}\xspace}
\newcommand{\R}{\ensuremath{\rm I\!R}\xspace}
\newcommand{\degr}{\ensuremath{^{\circ}}\xspace}
\newcommand{\iso}{\ensuremath{\textrm{iso}}\xspace}
\newcommand{\misid}{\ensuremath{\textrm{misid}}\xspace}
\newcommand{\gentauh}{\ensuremath{\textrm{\gen-\tauh}}\xspace}
\newcommand{\genjet}{\ensuremath{\textrm{\gen-jet}}\xspace}
\newcommand{\conjunction}{\ensuremath{\,\, \& \,\,}\xspace}
\newcommand{\cf}{cf.\xspace}
\newcommand{\efficiency}{\ensuremath{\varepsilon_{\Pgt}}\xspace}
\newcommand{\fakerate}{\ensuremath{P_{\textrm{misid}}}\xspace}
\newcommand{\X}{\ensuremath{\textrm{X}}\xspace}
\newcommand{\genX}{\ensuremath{\textrm{\gen-X}}\xspace}
\newcommand{\dxy}{\ensuremath{d_{\textrm{xy}}}\xspace}
\newcommand{\sigmaxy}{\ensuremath{\sigma_{d{\textrm{xy}}}}\xspace}
\newcommand{\dz}{\ensuremath{d_{\textrm{z}}}\xspace}
\newcommand{\sigmaz}{\ensuremath{\sigma_{d{\textrm{z}}}}\xspace}
\newcommand{\indicator}{\ensuremath{\mathcal{O}}\xspace}
\newcommand{\ch}{\ensuremath{\textrm{ch}}\xspace}
\newcommand{\nh}{\ensuremath{\textrm{nh}}\xspace}
\newcommand{\ig}{\ensuremath{\textrm{i}}\xspace}
\newcommand{\og}{\ensuremath{\textrm{o}}\xspace}
\DeclareMathOperator{\sign}{sgn}



\begin{document}
	\title{Reconstruction of transverse and longitudinal impact parameters with Key4HEP}
	\author[1]{Torben Lange}
	\affil[1]{National Institute of Chemical Physics and Biophysics (NICPB), R\"{a}vala pst 10, 10143 Tallinn, Estonia, torben.lange@cern.ch}
	\maketitle
	\begin{abstract}
		This short letter contains additional information for the $\tauh$ reconstruction algorithms found in \cite{christian_veelken_2023_8113344} and used in \cite{Lange:2023gbe}.
		Previous work on $\tauh$ identification has shown that the finite but non-zero lifetime of the $\Pgt$ lepton provides a handle to improve the discrimination of hadronic $\Pgt$ decays from quark and gluon jets. How these are computed in the context of the afformentioned software package and publication using the Key4HEP~\cite{Ganis:2021vgv} format is documented in the following.
	\end{abstract}
	
	The finite distance that a $\Pgt$ lepton travels between its production and decay ($c\tau = 87$~$\mu$m) causes the tracks of the $\hpm$ produced in the $\Pgt$ decay to be displaced from the primary event vertex (PV).
	The transverse ($\dxy$) and longitudinal ($\dz$) impact parameters quantify this displacement in the directions perpendicular and parallel to the beam axis.
	The $\dxy$ and $\dz$ are determined by the point of closest approach (PCA) between the track and the PV:
	\begin{eqnarray}
		\dxy & = & \sign\left(\left( \vec{r}_{\PCA} - \vec{r}_{\PV} \right) \cdot \vec{p}_{\jet}\right) \, \sqrt{ (x_{\PCA} - x_{\PV})^{2} + (y_{\PCA} - y_{\PV})^{2}} \nonumber \\
		\dz & = & \sign\left(\left( \vec{r}_{\PCA} - \vec{r}_{\PV} \right) \cdot \vec{p}_{\jet}\right) \, \vert z_{\PCA} - z_{\PV}) \vert \, ,
		\label{eq:ImpactParameters}
	\end{eqnarray}
	where $\sign$ refers to the signum function.
	Concerning the sign of $\dxy$ and $\dz$, we follow the convention used in Ref.~\cite{CMS:2011hta} and use a positive sign in case the momentum vector of the jet that seeds the $\tauh$ reconstruction points towards the same hemisphere as the difference between the PV and the PCA, and a negative sign if it points towards the opposite hemisphere.
	We denote the former by the symbol $\vec{p}_{\jet}$ and the latter by $\vec{r}_{\PCA} - \vec{r}_{\PV}$.
	The difference between the PV and the PCA represents an estimate of the $\Pgt$ flight direction.
	
	The $\dxy$ and $\dz$, and their respective uncertainties, $\sigmaxy$ and $\sigmaz$, are not part of the Key4HEP format used in Pandora and thus need to be computed for the work presented in this paper.
	We compute $\dxy$ and $\dz$ using the set of standard track parameters ($\phi_{0},\,\lambda,\,d_{0},\,z_{0},\,\Omega$) used in Key4HEP.
	The parameters $d_{0}$ and $z_{0}$ denote the distances between the track and the reference point in the $x$-$y$ plane and in $z$-direction, respectively; $\phi_{0}$ denotes the angle between the tangent to the track and the $x$-direction; 
	$\lambda= \cot\theta$ denotes the slope of the track in the $r$-$z$ plane; and $\Omega$ the inverse of the track's radius of curvature. Details on the definition of these parameters can be found in Ref.~\cite{Kramer:2006zz}. The trajectory of charged particles propagating through the solenoidal magnetic field within the detector has the form of a helix. Computing the PCA of the helix with respect to the PV is a non-trivial task, because the relevant equations are non-linear.
	We simplify this task by taking advantage of the fact that the distances that $\Pgt$ leptons travel between their production and decay are typically small compared to the expected radius of curvature of the tracks produced in the $\Pgt$ decay, which allows us to use a linear approximation of the helix equations.
	
	To obtain the expressions for the linear approximation, we start with the parametrisation of the helix given in Ref.~\cite{Kuhr:1998jk}:
	\begin{equation}
		\vec{r}_{\trk}(s) = \begin{pmatrix}
			x_c + 1/\Omega \, \sin(\phi_0 + s \, \Omega) \\
			y_c - 1/\Omega \, \cos(\phi_0 + s \, \Omega) \\
			z_c + s \, \lambda
		\end{pmatrix} 
		\, ,
		\label{eq:helixtrk1}
	\end{equation}
	where $\vec{r}_{\trk}(s)$ denotes a point on the helix and $s$ the travel distance of the charged particle in the transverse plane.
	To obtain a linear approximation of this equation we perform a Taylor expansion with respect to the travel distance $s$ at the point $s = 0$ and keep the constant term and the term linear in $s$. This yields:
	\begin{equation}
		\vec{r}_{\trk}(s) = \begin{pmatrix}
			s \, \cos(\phi_0) + x_{c} + 1/\Omega \, \sin(\phi_0) \\
			s \, \sin(\phi_0) + y_{c} - 1/\Omega \, \cos(\phi_0) \\
			z_c + s \, \lambda
		\end{pmatrix} 
		\, .
		\label{eq:lineartrk}
	\end{equation} 
The parametrisations in Eqs.~(\ref{eq:helixtrk1}) and~(\ref{eq:lineartrk}) use the centre, $\vec{r}_{c} = (x_{c},\,y_{c},\,z_{c})$, of the helix trajectory in the $x$-$y$ plane as the reference point.
The track parameters ($\phi_{0},\,\lambda,\,d_{0},\,z_{0},\,\Omega$) are available for different reference points $\vec{r}_{\refpt} = (x_{\refpt},\,y_{\refpt},\,z_{\refpt})$ in Key4HEP. Among the available $\vec{r}_{\refpt}$, we choose the one that we expect to be closest to the PV: the reference point coinciding with the nominal interaction point at the centre of the detector, $\vec{r}_{\refpt} = (0,\,0,\,0)$.
We follow the convention of Ref.~\cite{Kramer:2006zz} and change the parametrisation such that it refers to a general reference point $\vec{r}_{\refpt} = (x_{\refpt},\,y_{\refpt},\,z_{\refpt})$. By definition of the track parameters $\phi_{0}$, $d_{0}$, and $z_{0}$ the position of the point on the helix at the travel distance $s=0$ is given by:
\begin{equation}
\vec{r}_{\trk}(s=0) = \begin{pmatrix}
	x_{\refpt} + \cos(\frac{\pi}{2} - \phi_0) \, d_0 \\
	y_{\refpt} - \sin(\frac{\pi}{2} - \phi_0) \, d_0 \\
	z_{\refpt} + z_0
\end{pmatrix} 
\label{eq:trkats0}
\end{equation}  
By comparing Eqs.~(\ref{eq:lineartrk}) and ~(\ref{eq:trkats0}), we find that the linear parametrisation with respect to the reference point $\vec{r}_{\refpt}$ is given by the expression:
\begin{equation}
\vec{r}_{\trk}(s) = \begin{pmatrix}
	s \, \cos(\phi_0) + x_{\refpt} + \cos(\frac{\pi}{2} - \phi_0) \, d_0 \\
	s \, \sin(\phi_0) + y_{\refpt} - \sin(\frac{\pi}{2} - \phi_0) \, d_0 \\
	s \, \lambda + z_{\refpt} + z_0 \, .
\end{pmatrix} 
\label{eq:lineartrk2}
\end{equation}    
We rewrite Eq.~(\ref{eq:lineartrk2}) in the form:
\begin{equation}
\vec{r}_{\trk}(s) = \vec{r}_{a} + \left(\vec{r}_{b} - \vec{r}_{a}\right) \, s \, ,
\label{eq:lineartrk3}
\end{equation}
with:
\begin{equation}
\vec{r}_{a} = \begin{pmatrix}
	x_{\refpt} + \cos(\frac{\pi}{2} - \phi_0) \, d_0 \\
	y_{\refpt} - \sin(\frac{\pi}{2} - \phi_0) \, d_0 \\
	z_{\refpt} + z_0
\end{pmatrix}
\quad \textrm{and} \quad 
\vec{r}_{b} - \vec{r}_{a} = \begin{pmatrix}
	\cos(\phi_0) \\
	 \sin(\phi_0) \\
	\lambda 
\end{pmatrix}
\, .
\end{equation}    
This is the equation of a straight line in three dimensions.

Using this linear approximation, the PCA between the track and the PV is given by the following expression for the travel distance $s$~\cite{weissteinWolfram}:
\begin{equation}
s_{\PCA} = -\frac{\left(\vec{r}_a - \vec{r}_{\PV}\right) \cdot \left(\vec{r}_{b} - \vec{r}_{a}\right)}{\vert\vec{r}_{b} - \vec{r}_{a}\vert^{2}} \, .
\label{eq:pca3d}  
\end{equation}
The above expression holds in three dimensions.
We restrict the computation of the PCA between the track and the PV to the $x$-$y$ plane. The corresponding expression for the travel distance $s$ in two dimensions is:
\begin{equation}
s_{\PCA} = -\frac{\left(x_{a} - x_{\PV}\right) \cdot \left(x_{b} - x_{a} \right) + \left(y_{a} - y_{\PV}\right) \cdot \left(y_{b} - y_{a}\right)}{\left(x_{b} - x_{a}\right)^{2} + \left(y_{b} - y_{a}\right)^{2}} \, .
\label{eq:pca2d}
\end{equation}
Inserting Eq.~(\ref{eq:pca2d}) into Eq.~(\ref{eq:lineartrk3}), we obtain the location of the PCA: $\vec{r}_{\PCA}=\vec{r}_{\trk}(s_{\PCA})$, which allows us to compute the transverse and longitudinal impact parameters by means of Eq.~(\ref{eq:ImpactParameters}). The uncertainties $\sigmaxy$ and $\sigmaz$ are computed by standard error propagation~\cite{Cowan:1998ji}, assuming the uncertainties on the five track parameters ($\phi_{0},\,\lambda,\,d_{0},\,z_{0},\,\Omega$) to be uncorrelated.
\section*{Acknowledgements}
The work presented here was performed in the context of the paper in \cite{Lange:2023gbe} and profited from fruitfull discussions with all of its authors who also contributed to the review and editing of this text. Therefore I would like to thank Saswati Nandan, Joosep Pataa, Laurits Tani and Christian Veelken for their contributions.
\clearpage
\bibliography{impact}
\end{document}
